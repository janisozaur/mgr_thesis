%*****************************************
\chapter{Wstęp}\label{ch:introduction}
%*****************************************

System śledzenia ruchów to pojęcie, które obejmuje szeroką gamę zestawów składających się z urządzeń dostarczających danych o położeniu oraz wyspecjalizowanego oprogramowania, które te informacje przetwarza.
Systemy takie, w ogólnym ujęciu, można spotkać w wielu miejscach:
\begin{aenumerate}
  \item w fabrykach śledzone jest przemieszczanie się produktów w celu ich dystrybucji,
  \item zarówno w produkcji gier jak i filmów stosuje się systemy \textsl{motion capture}, czyli przechwytywania ruchów,
  \item systemem śledzenia ruchów można nazwać także fotoradar przy drodze,
  \item cyfrowe modelowanie, w celu odtworzenia wirtualnego modelu istniejącego przedmiotu, często posługuje się systemem śledzenia ruchów pod postacią digitizera.
\end{aenumerate}

W każdym przypadku zarówno metoda pozyskiwania danych, protokół, którym są przekazywane oraz oprogramowanie te dane odbierające i przetwarzające jest inne, gdyż inne są wymagania, założenia i~konstrukcja każdego z tych systemów.

Spośród mnogiej ilości zastosowań takich systemów najbardziej interesującym z mojego punktu widzenia jest śledzenie ruchów na potrzeby odwzorowania ich w grach wideo.

W swojej pracy przybliżę istniejące systemy wykorzystywane w konsolach w dniu dzisiejszym oraz zaprezentuję metodę śledzenia ruchów, którą można wykorzystać do interakcji z komputerem.
