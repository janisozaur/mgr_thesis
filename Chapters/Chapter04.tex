%************************************************
\myChapter{Sprzęt}\label{ch:hardware}
%************************************************

Podstawowym elementem pracy było stworzenie sprzętu - urządzenia wejściowego, które będzie pozwalało na śledzenie ruchów palców użytkownika.

Podczas projektowania powstało kilka dodatkowych założeń, które sprzęt powinien spełniać:
\begin{itemize}
 \item łatwość podłączenia - urządzenie powinno być możliwe do ,,zainstalowania'' w kilka chwil, najlepiej bez konieczności korzystania z narzędzi, podłączania zbyt wielu przewodów,
 \item łatwość użytkowania - urządzenie powinno umożliwiać interakcję z komputerem osobom niezapoznanym z pracą, niekoniecznie posiadającym wiedzę techniczną dotyczącą zasady działania,
 \item działanie w możliwie najróżniejszych warunkach - uwzględnienie spodziewanych warunków pracy urządzenia, tak aby nie przeszkadzały one w jego użytkowaniu,
 \item prostota i modułowość - zmniejszenie kosztów produkcji poprzez zastosowanie możliwie prostych, łatwo dostępnych i sprawdzonych części połączonych w łatwo wymienialne moduły, co dodatkowo poprawi możliwość zastępowania części w wypadku ich uszkodzenia,
\end{itemize}

Większość postawionych celów udało się zrealizować.

Urządzenie składa się z aluminiowej ramy, złożonej z czterech kątowników, z wywierconymi otworami na diody, fotodiody i elementy mocowania. Rama została zaprojektowana, aby pasowała do posiadanego monitora o przekątnej 24\". Do ramy przymocowanych jest 20 modułów - po 4 na krótsze i po 6 na dłuższe boki.

Każdy z jednakowych modułów zawiera:
\begin{itemize}
 \item diodę emitującą światło w paśmie podczerwieni (TSAL6400),
 \item 8 fotodiod reagujących na światło wymienionej wyżej diody,
 \item ośmioportowy trzystanowy bufor (74LS541),
 \item zestaw złącz, oporników wymaganych do funkcjonowania powyższych elementów.
\end{itemize}

Moduły połączone zostały 8-kanałowym interfejsem równoległym, dodatkowo z dwiema żyłami niosącymi zasilanie oraz osobnymi przewodami z sygnałami chip select (\textsmaller{\textoverline{CS}}) dla bufora oraz diody świecącej (\textsmaller{CS}).

Dwie płytki drukowane, każda z zestawem układów 74HC4514 i 74HC4515, pełnią funkcję multipleksera portów - układy te to dekodery 4-do-16 z zapadkami (\textsl{latches}), które umożliwiają komunikację z modułami za pomocą znacznie zredukowanej ilości pinów nić byłaby wymagana, gdyby każdy moduł podłączać bezpośrednio. Zastosowanie takiego rozwiązania jest szybsze niż w przypadku interfejsu szeregowego, np. takiego jak I\textsuperscript{2}C lub SPI, jednak pociąga za sobą konieczność poprowadzenia sygnału \textsmaller{CS} oraz \textsmaller{\textoverline{CS}} do każdego modułu osobno zamiast zintegrowania go w magistrali łączącej wszystkie moduły wygodną do podłączania tasiemką. Konieczność instalowania w każdym module dekodera takiego interfejsu, ustawiania adresów logicznych modułów, chociaż pozwalałaby na dalsze rozszerzanie ramki o kolejne moduły, zwiększałaby koszty oraz wymagany nakład pracy, rozumianej zarówno przez tworzone oprogramowanie jak i ,,koszt'' działania programu, tj. czasu wymaganego do
obsługi wybranego protokołu, zaważyły o ostatecznym wyborze rozwiązania równoległego.

Wszystkie płytki PCB, z wyjątkiem modułu mikrokontrolera, zostały zaprojektowane i wykonane wyłącznie w celu realizacji pracy.

Stworzony sprzęt przeszedł kilka zmian: zarówno mikroprocesor jak i elementy konstrukcji były kilkukrotnie wymieniane.\\

\section{Rewizja pierwsza}

W pierwszej wersji wykorzystany został mikrokontroler z 16-bitowej rodziny MSP430: MSP430FG4618 znajdujący się w płytce Experimenter's Board. Pod kątem tego urządzenia projektowane były wszystkie interfejsy, uwzględniając przede wszystkim ilość dostępnych pinów oraz charakterystykę prądową.

Niedopatrzeniem okazał się moduł komunikacji szeregowej UART, który operował w standardzie RS232/TTL\graffito{transistor-to-transistor logic, poziomy napięć i prądów przystosowane do niedługich połączeń pomiędzy układami}. Po podłączeniu adaptera USB-RS232 opartego o renomowany układ FTDI FT2232C okazało się w testach, że podczas dużego obciążenia hosta, pojedyncze bity ramki RS232 mogą się ,,zgubić'', tj.
jedynki sporadycznie zastępowane były zerami. Ze względu na spodziewany charakter wykorzystywania pracy, np. podczas gier, oczekiwane było poprawne działanie także w warunkach wysokiego obciążenia, co dyskwalifikowało takie rozwiązanie.\\

\section{Rewizja druga}

Wymaganą stabilność transmisji danych miał zapewnić układ z tej samej rodziny, MSP430F2274 zawarty w płytce eZ430-RF2500. Płytka ta podłączana była do komputera bezpośrednio do portu USB. Testy szybko pokazały, że w urządzeniu brakuje dostatecznej ilości możliwych do wykorzystania pinów, pomimo stworzenia kolejnej wersji płytki interfejsu ramka $\leftrightarrow$ mikrokontroler, która ograniczała liczbę wymaganych połączeń z 24 do 20 - możliwie najniższej ilości niewymagającej połączeń szeregowych. Ponadto prędkość transmisji była sztucznie ograniczona do wartości 9600BPS, która nie zapewnia wymaganej przepustowości.\\

\section{Rewizja trzecia}

Ze względu na problemy z komunikacją, zdecydowałem się wykorzystać do pracy nowy układ z rdzeniem ARM Cortex-M4F, Texas Instruments LM4F120H5QR na płytce Stellaris Launchpad. Jest to 32-bitowy procesor, dostępny dopiero od listopada 2012, tj. kilka miesięcy po rozpoczęciu pisania pracy.\\

\section{Rewizja czwarta}

W tej wersji, zmieniona została konstrukcja samej ramki i modułów. Pierwotny projekt zakładał, że diody emitujące światło znajdować będą się w innej płaszczyżnie niż elemnty odbiorcze w celu uniknięcia problemu odbić. Moc wykorzystanych wtedy diod okazała się niewystarczająca do poprawnego oświetlenia odbiorników, przesunięte zostały więc one do płaszczyzny odbiorników.

Z powodu nadal niedostatecznej ilości mocy diod, wymienione zostały na modele 5\texttimes\  mocniejsze, używające 120mA: TSAL6400. Dodane także zostały (do niektórych modułów) diody świecące w paśmie światła widzialnego, element nieprzewidziany w pierwotnym projekcie, co pozwoliło na prostsze debugowanie działania urządzenia.\\

\section{Interfejs ramka $\leftrightarrow$ mikrokontroler}

Interfejs zapewniający komunikację pomiędzy mikrokontrolerem, a ramką przeszedł tylko jedną wymianę, mającą na celu przede wszystkim wspomniane już ograniczenie ilości wymaganych pinów do obsługi.

Pierwsza wersja oparta była na układach 74LS139 (podwójny selektor 1-z-4) oraz 74LS14 (inwerter), posiadała także zestaw diod LED i oporników pomocnych w debugowaniu. Działanie tego urządzenia było bardzo proste, polegało na wyborze odpowiedniego modułu (nadawczego i odbiorczego) poprzez ustawienie sygnałów \textsmaller{CS} oraz \textsmaller{\textoverline{CS}} po wybraniu danego układu oraz ,,wysłania'' do niego adresu jednej z czterech linii wyjściowych. Sygnał \textsmaller{\textoverline{CS}} uzyskany został przez zastosowanie inwertera 74LS14.

Druga, a zarazem finalna wersja tego sprzętu wykorzystuje w tym celu dwie płytki PCB, każda zawierająca po jednym układzie 74HC4514 oraz 74HC4515. Układy te to selektory 1-z-16, które różnią się tylko i wyłącznie wbudowanym inwerterem: 74HC4514 posiada wolną jedynkę, tzn. na wybranym wyjściu ustawiony zostanie stan wysoki, pozostałe linie mają stan niski; układ 74HC4515 posiada wolne zero - na każdej linii znajduje się inwerter.\\

Zastosowanie dwóch selektorów 16-liniowych wymaga 6-ciu linii: 4 linie adresowe oraz po jednej linii \textsmaller{\textoverline{I}} (\textsmaller{\textoverline{Inhibit}}) dla każdego układu. Chociaż 20 użytych modułów można zaadresować za pomocą 5-bitowej przestrzeni adresowej zdolnej do obsługi maksymalnie 32 odbiorników, to należy zwrócić uwagę, że dekodery 5-do-32 nie istnieją. Ich konstrukcja, choć trywialna, dodawałaby zbędnego skomplikowania oraz kosztów, dając w zamian niewiele.
Dodatkowo, posiadanie dedykowanych linii wybierających pozwala na przełączenie urządzenia w tryb ,,stand-by'', co w tym wypadku oznacza wyłączenie zarówno nadawania jak i odbierania. Podwójne elementy sterujące umożliwiają także ograniczenie długości potrzebnych przewodów, rozmiarów płytki (ze względu na zmniejszoną ilość wyjść) oraz konieczność mniejszej ingerencji w urządzenie w razie awarii.