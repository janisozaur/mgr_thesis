%*****************************************
\chapter{Wyniki}\label{ch:results}
%*****************************************

Przedstawione w poprzednim rozdziale rezultaty testów pokazują zachowanie algorytmu i sprzętu w różnych przypadkach.

W szczególności rysunki \ref{fig:scene_empty_1}, \ref{fig:scene_empty_2}, \ref{fig:scene_with_one_obstacle_1} i \ref{fig:scene_with_one_obstacle_2} prezentują możliwą do uzyskania dokładność.

Pierwsza para obrazuje poziom szumów w zwracanych danych.
Jest on miejscami dostatecznie duży, aby mógł zostać pomylony z prawdziwym obiektem znajdującym się w ramce.\\

Druga para obrazków pokazuje wielkość obiektu, jaki potrzebny był do testów.

Obiekty mniejsze, a w szczególności palce były dla systemu ledwo widoczne lub zupełnie niewidoczne.

Nawet w przypadku tak dużych obiektów, zwracane dane nie były zadowalającej jakości, posiadały wiele przekłamań, które ograniczały możliwość prawidłowego wykrycia konturu przeszkody.\\

Zaimplementowany algorytm mógłby dawać dobre rezultaty, gdyby nie znaczny poziom szumu danych wejściowych.

Jest on jednocześnie prosty w implementacji, oparty o proste działania, musi jednak przetwarzać duże ilości danych (w zależności od ustawionych opcji).
