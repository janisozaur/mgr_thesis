%************************************************
\myChapter{Oprogramowanie}\label{ch:software}
%************************************************

Jako zwolennik wolnego i otwartego oprogramowania starałem się korzystać tylko i wyłącznie z takich właśnie narzędzi.

\graffito{ \includegraphics[width=\marginparwidth]{gfx/gnuhead_inkscape.pdf} Logo Fundacji Wolnego Oprogramowania \ppauza Free Software Foundation \citep{FSF}}Sama praca udostępniana jest na zasadach otwartej licencji \textsc{GNU GPL 3.0}. Wszystkie otwarte licencje typu \textsc{GNU GPL} stworzone zostały przez fundację FSF, która też stoi na straży ich przestrzegania.

Oprogramowanie stworzone na potrzeby pracy składa się z dwóch elementów:
\begin{itemize}
 \item oprogramowania na komputer,
 \item oprogramowania na mikrokontroler.\\
\end{itemize}

Większość zmian mikrokontrolera, opisywana w rozdziale \ref{ch:hardware}, pociągała za sobą przepisanie oprogramowania praktycznie od zera ze względu na rdzenie z kompletnie innych rodzin, inne dostępne moduły oraz inny sposób ich obsługi. Ilość tych zmian, a także dodatkowe utrudnienia pod postacią niedostępności debugowania, skomplikowanej instalacji słabo dostępnych i przestarzałych pakietów kompilatora skłoniły mnie do poszukania rozwiązania typu RTOS lub podobnego.
W taki sposób znalazłem środowisko \textsc{Energia}, \textit{fork} środowiska \textsc{Arduino}\graffito{Programy w środowisku \textsc{Arduino} noszą nazwę ,,sketch''.}, oparty o toolchain \textsc{msp-gcc} oraz \textsc{arm-gcc}, pozwalający pisać programy w języku C++ ukrywając wiele szczegółowych operacji przed programistą jak np. konfiguracja stosu USB, pozwalając mu skoncentrować się na istocie tworzonego programu.\\

\section{Protokół komunikacji}

Komunikacja hosta z mikrokontrolerem odbywa się przez wirtualny port szeregowy (urządzenie \textsc{USB Communications Device Class, CDC}). Narzuca to kilka ograniczeń dotyczących możliwości protokołu, które należy uwzględnić podczas jego projektowania:
\begin{itemize}
 \item mikrokontroler będzie widoczny jako zwykłe urządzenie znakowe, należy więc zadbać, aby przesyłane dane były w postaci czytelnej także w zwykłym kliencie terminala,
 \item sposób komunikacji musi uwzględniać możliwość synchronizacji, np. za pomocą specjalnego znaku, który nie pojawia się w innym przypadku,
 \item protokół musi być możliwie zwięzły ze względu na ograniczoną przepustowość łącza,
 \item komunikacja musi przebiegać dwukierunkowo (w trybie przynajmniej half-duplex).\\
\end{itemize}

Uwzględniając powyższe wytyczne, opracowałem protokół o następujących cechach:
\begin{itemize}
 \item Wszystkie przesyłane dane są pod postacią znaków kodu ASCII, zarówno z hosta do mikrokontrolera jak i z powrotem. Używane są trzy sposoby tłumaczenia danych z postaci binarnej na kod ASCII:
 \begin{enumerate}
  \item Numeryczny zapis dziesiętny \pauza wartości zapisywane są w~systemie dziesiętnym po jednej stronie, druga strona konwertuje liczbę ponownie na system binarny.
  \item Base64\label{item:base64} \pauza System konwersji danych konwertujący ciąg 8-bitowych danych na ciąg 6-bitowych danych z określonym alfabetem. Algorytm ten opisany jest w sekcji~\ref{sec:base64}.
  \item Zapis binarny z offsetem \pauza dla ograniczonych wielkości zmiennych zastosowałem przesunięcie wartości, które zapewnia ich prezentację w kodzie ASCII: do wartości zmiennej dodawany jest offset 'a', tak więc 'a' rozumiane jest jako 0, 'b' jako 1, itd.
 \end{enumerate}
 \item Do synchronizacji wykorzystany został znak o kodzie \textsc{0x0D}, czyli powrót karetki (\textit{carriage return}), dzięki czemu synchronizacja jest czytelnia zarówno dla komputera jak i człowieka.
 \item Protokół został zaprojektowany w modelu klient-serwer, gdzie klient (komputer) odpytuje o dane serwer (mikrokontroler).\\
\end{itemize}

O wyborze algorytmu Base64 (\ref{item:base64}) zadecydowała jego powszechność, co przekłada się na dostępność narzędzi kodujących i dekodujących.
Biblioteki \texttt{Qt} także dostarczają implementacji funkcji obsługujących to kodowanie.

Prostota tego algorytmu przekłada się na możliwość stworzenia zoptymalizowanej implementacji, która na mikrokontrolerze może znacząco się przekładać na szybkość transmisji danych.

Dodatkowo, dla pewnych struktur danych, nawet w postaci zakodowanej można szybko ocenić czy są one poprawne.\\

Dzięki identyfikowaniu się urządzenia jako port szeregowy, możliwe jest podłączenie się dowolnym klientem terminala (np. \textsc{screen}, \textsc{minicom}, itp.) i ręczna obsługa urządzenia.
Znacznie pomaga to podczas rozwijania oprogramowania umożliwiając łatwe wprowadzenie systemu w~skrajne przypadki.\\

Obsługę protokołu od strony mikrokontrolera można przedstawić pseudokodem z algorytmów~\ref{alg:proto_uc_interrupt}~i~\ref{alg:proto_uc_main}.
\begin{algorithm}
\caption{Obsługa protokołu komunikacji, strona mikrokontrolera, przerwanie portu szeregowego}
\label{alg:proto_uc_interrupt}
\begin{algorithmic}[1]
  \REQUIRE \texttt{dane} \ppauza tablica statyczna, do której końca dopisywane są przychodzące dane. Wystarczająco duża na pomieszczenie spodziewanej ilości danych.\\
  \texttt{Serial} \ppauza obiekt klasy obsługującej port szeregowy\\
  \texttt{input, length, commandReady} \ppauza zmienne wykorzystywane do synchronizacji pomiędzy obsługą przerwania, a główną pętlą programu
  \STATE count \textleftarrow{} $0$
  \WHILE{Serial.available()}
    \IF{commandReady == true}
      \STATE continue;
    \ENDIF
    \STATE char \textleftarrow{} Serial.read()
    \STATE dane[count++] \textleftarrow{} char
    \IF{char == '\textbackslash{}r'}
      \STATE length \textleftarrow{} count
      \STATE input \textleftarrow{} dane
      \STATE count \textleftarrow{} $0$
      \STATE commandReady \textleftarrow{} true
    \ENDIF
  \ENDWHILE
\end{algorithmic}
\end{algorithm}

\begin{algorithm}
\caption{Obsługa protokołu komunikacji, strona mikrokontrolera, główna pętla}
\label{alg:proto_uc_main}
\begin{algorithmic}[1]
  \REQUIRE \texttt{Serial} \ppauza obiekt klasy obsługującej port szeregowy\\
  \texttt{input, length, commandReady} \ppauza zmienne wykorzystywane do synchronizacji pomiędzy obsługą przerwania, a główną pętlą programu\\
  \texttt{toBase64(input, output, inputLength)} \ppauza funkcja zamieniająca ciąg danych \texttt{input} o długości \texttt{inputLength} na kodowanie Base64 i zapisująca wynik do \texttt{output}, zwracająca długość ciągu \texttt{output}\\
  \texttt{handleCommand(input, output)} \ppauza funkcja obsługi komend protokołu, zapisuje ewentualne dane zwrotne w buforze \texttt{output}, zwraca długość zapisanych danych
  \WHILE{true}
    \IF{commandReady == true}
      \STATE resultLength \textleftarrow{} handleCommand(input, commandBuffer)
      \STATE b64Length \textleftarrow{} toBase64(commandBuffer, outputBuffer, resultLength)
      \STATE Serial.write(outputBuffer, b64Length)
      \STATE Serial.write('\textbackslash{}r')
      \STATE commandReady \textleftarrow{} false
    \ENDIF
  \ENDWHILE
\end{algorithmic}
\end{algorithm}

\section{Oprogramowanie na komputer}
W celu stworzenia oprogramowania dla komputera wykorzystałem kilka środowisk i bibliotek.

\paragraph{Qt}
Wieloplatformowe środowisko (ang. \textsl{framework}) \textsmaller{Qt}, licencjonowane wolną licencją \textsc{GNU LGPL 2.1} oraz \textsc{GNU GPL 3.0}, składa się z kilku komponentów.
W jego skład wchodzą między innymi: biblioteka \textsc{Qt} oraz kompilator \verb|moc|.
Wszystkie te elementy znacznie usprawniają pisanie aplikacji w języku \verb|C++| dostarczając metod, które abstrahują od specyfiki wykorzystywanego systemu operacyjnego.

Pisanie nawet skomplikowanych programów z wykorzystaniem \textsc{Qt} jest proste, łatwe, szybkie i przyjemne.
Programista świadomy różnic pomiędzy systemami operacyjnymi i delegujący obsługę parametrów do metod dostarczanych przez klasy bibliotek \textsc{Qt} zyskuje możliwość skompilowania swojego kodu pod platformy \textsc{Linux}, \textsc{Windows} oraz \textsc{Mac OS X} bez konieczności jakichkolwiek zmian.

Szerokie spektrum dostarczanych klas (począwszy od kontenerów danych i metod iteracji, przez obsługę urządzeń wejściowych, przez obsługę sieci, aż po rysowanie i zarządzanie grafikami i wiele, wiele innych\ldots) zapewnia, że do zaimplementowania wielu aplikacji nie będzie wymagane wykorzystanie bibliotek trzecich.
\newline
\newline
\textsl{Sygnały i sloty}
Ze względu na intensywne wykorzystywanie w stworzonych aplikacjach połączeń sygnał-slot, zamieszczam poniżej ich opis.

Centralną właściwością środowiska \textsc{Qt}, a jednocześnie jedną z najbardziej odróżniających ten framework od innych, jest mechanizm sygnałów i slotów będący zarządzaną metodą komunikacji pomiędzy obiektami.

Chociaż na pierwszy rzut oka przypomina ona znane dotychczas metody oparte o wywołania zwrotne (ang. \textsl{callback}) i faktycznie się z~nich wywodzi, to jednak różni się od nich w kilku kluczowych aspektach.

Jest to metoda dynamiczna (działająca w czasie wykonywania \ppauza ang. \textsl{runtime}).
Metoda wywołań zwrotnych wykorzystywana jest głównie do zamodelowania statycznych (tj. znanych już podczas czasu kompilacji \ppauza ang. \textsl{compiletime}) powiązań pomiędzy obiektami jak np:
\begin{verse}
użytkownik kliknął przycisk \texttimes~\textrightarrow~wywołaj metodę \texttt{close()}
\end{verse}

W przypadku tym przypisuje się wskaźnik do funkcji do pewnego pola klasy wywołującej, który w momencie zajścia zdarzenia jest wywoływany.
Konsekwencją takiego podejścia jest konieczność znania odbiorników w czasie kompilacji, co uniemożliwia np. ładowanie wtyczek w czasie rzeczywistym i powiadamiania ich o zdarzeniach w następujący sposób:
\begin{verse}
powiadom wszystkie odbiorniki o zajściu zdarzenia $\omega$
\end{verse}

Zastosowanie w tym miejscu wektorów wskaźników jest jedynie obejściem problemu, a nie jego rozwiązaniem, gdyż nakłada na programistę obowiązek pamiętania o poprawnym przydzielaniu i zwalnianiu pamięci na te elementy, w przypadku aplikacji wielowątkowych szczegółowego analizowania zależności czasowych pomiędzy wywołaniami kodu obiektów, oraz drobiazgowego sprawdzania typów wywołań.

Podejście to wprowadza ponadto nadmiarową i zbędną wiedzę o~odbiornikach do nadajnika.

Rozwiązanie dostarczane przez framework \textsc{Qt} jest elegancką metodą pozbycia się wymienionych wad na rzecz udostępnienia programiście prostego w obsłudze, lecz potężnego i skutecznego, mechanizmu dynamicznego wiązania obiektów w pary nadajnik-odbiornik.

Przekazywane sygnały są ,,wątkowo-bezpieczne'' (ang. \textsl{thread-safe}), co pozwala na przetwarzanie sygnału w innym wątku niż ten, który zainicjował jego wysłanie.
Aby zrealizować takie podejście, każdy z~obiektów dziedziczących po klasie \verb|QObject| powinien należeć do jakiegoś wątku, tak aby był on uwzględniany w pętli zdarzeń (ang. \textsl{event loop}), jaka jest przez ten wątek przetwarzana.
Domyślnie każdy nowy obiekt przynależy do wątku rodzica, który go utworzył.

Pociąga to za sobą konsekwencję, że jeżeli obiekt nie przynależy do żadnego wątku, to jego zdarzenia nie będą przetwarzane.

W celu uproszczenia obsługi, środowisko \textsc{Qt} domyślnie wykorzystuje do wszystkich obiektów wątek główny aplikacji i jeśli programista jawnie nie usunie z niego obiektów, to ich sygnały będą przetwarzane właśnie w tym wątku.

Aby zintegrować to rozwiązanie z kodem, postanowiono ,,rozszerzyć'' standard\graffito{,,Rozszerzenie'' to wykorzystywane jest tylko i wyłącznie do środowiskia \textsc{Qt}.} języka \verb|C++|.
Do istniejących już kwalifikatorów \verb|public|, \verb|private| i~\verb|protected| dodano dwa nowe: \verb|signals| i \verb|slots|.

Kwalifikator \verb|signals| definiuje sygnały. Mają one taką samą strukturę, jak zwykłe metody klasy z następującymi wyjątkami:
\begin{aenumerate}
  \item definicja sygnału jest tylko i wyłącznie jego deklaracją, oznacza to, że sygnał nie posiada żadnego kodu, jest tylko abstrakcyjnym tworem komunikującem zajście pewnego zdarzenia,
  \item sygnały nie mogą zwracać wartości \ppauza zwracanym typem musi być \verb|void|, a wszelkie przekazywane dane zawarte są w argumentach,
  \item sygnały są zawsze publiczne.
\end{aenumerate}

Deklarację pewnej minimalnej klasy zawierającej sygnały prezentuje listing \ref{lst:signals_declaration}.

\begin{listing}
  \lstinputlisting{listings/signals_declaration.cpp}
  \caption{Klasa zawierająca sygnały}
  \label{lst:signals_declaration}
\end{listing}

Jak pokazano, aby klasa mogła wykorzystać mechanizm sygnałów (a także slotów), musi ona dziedziczyć z klasy \verb|QObject| i wywoływać makro \verb|Q_OBJECT| w prywatnej części deklaracji.

Kwalifikator \verb|slots|, jak łatwo się domyślić, deklaruje sloty.
Poza dodatkową możliwością wywołania slotu, są to tradycyjne metody klasy i tak samo obowiązują je pozostałe kwalifikatory: \verb|public|, \verb|private| i~\verb|protected|.
W odróżnieniu od sygnałów, sloty wymagają dostarczenia implementacji i mogą zwracać wartości.

Deklarację pewnej minimalnej klasy zawierającej sloty prezentuje listing \ref{lst:slots_declaration}.

\begin{listing}
  \lstinputlisting{listings/slots_declaration.cpp}
  \caption{Klasa zawierająca sloty}
  \label{lst:slots_declaration}
\end{listing}

Połączenie emitera ze słuchaczem następuje poprzez wywołanie metody \verb|connect| klasy \verb|QObject|, której argumentami są obiekty nadający i~odbierający, a także nazwy łączonych metod.
Sygnatura slotu i sygnału musi być taka sama, z wyjątkiem zwracanego typu.

Użyteczną cechą jest możliwość podłączenia sygnału do sygnału, dzięki czemu zostanie wywołany drugi z sygnałów, a w efekcie podłączone do niego sloty.

Pozostała charakterystyka tego rozwiązania, taka jak przekazywanie meta-typów, rozgraniczenie pomiędzy obiektem nadającym, a odbierającym oraz inne, nie została wykorzystana w stworzonym oprogramowaniu, w związku z czym odsyłam czytelnika do dokumentacji środowiska \textsc{Qt} \citep{Qt}.
\newline
\newline
\textsl{Kompilator moc} Ponieważ powyższe rozwiązanie nie należy do standardu języka \verb|C++|, zaś całe oprogramowanie stworzone przy pomocy środowiska \textsc{Qt} kompilowane jest przy pomocy kompilatora \verb|C++| zgodnego ze standardem ISO/IEC C++\citep{CPPStandard}, takiego jak \texttt{g++}, dostarczany jest wraz z \textsc{Qt} kompilator \texttt{moc}, czyli \textsl{meta-object compiler}.

Zadaniem tego narzędzia jest przeparsowanie dostarczonych plików źródłowych pod kątem odszukania wśród nich deklaracji klas dziedziczących z \verb|QObject|, a zatem wykorzystujących rozszerzone możliwości oferowane przez \textsc{Qt} i wygenerowanie kodu zrozumiałego przez wspomniany wyżej ,,zwykły'' kompilator.
Kod dostarczony przez programistę, pozbawiony rozszerzeń \textsc{Qt} oraz kod wygenerowany przez narzędzie \texttt{moc} są kompilowane, a następnie łączone ze sobą na etapie linkowania.

\paragraph{QSerialPort}
Biblioteka \textsc{QSerialPort}, oparta o wolną licencję \textsc{GNU LGPL 2.1}, dostarcza metod komunikacji wykorzystujących port szeregowy.
Opiera się ona o środowisko \textsc{Qt}, przez co bardzo łatwo jest zintegrować ją z projektami korzystającymi z tych narzędzi.
Podobnie jak samo \textsc{Qt}, biblioteka ta jest wieloplatformowa, co było jednym z~głównych powodów, dla których wybrałem właśnie ją.

Dostarcza ona metody obsługi portów oparte o interfejs \verb|QIODevice| udostępniany przez \textsc{Qt}, które implementowane są z wykorzystaniem natywnych funkcji systemowych dla każdej z platform, dzięki czemu przekazywanie danych odbywa się szybko i sprawnie \citep{QSP}.
