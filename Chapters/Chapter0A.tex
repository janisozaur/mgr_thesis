%********************************************************************
% Appendix
%*******************************************************
% If problems with the headers: get headings in appendix etc. right
%\markboth{\spacedlowsmallcaps{Appendix}}{\spacedlowsmallcaps{Appendix}}
\chapter{Załącznik}

\section{Kodowanie Base64}\label{sec:base64}

Kodowanie Base64 jest formą zapisu danych ośmiobitowych za pomocą znaków z sześciobitowego alfabetu $A$. Zwykle alfabet, na który dokonywane jest takie mapowanie, składa się z wielkich i małych liter alfabetu angielskiego (26 + 26 = 52 znaki), dziesięciu cyfr, dwóch znaków dodatkowych: '+', '/', w sumie $2^6 = 64$ znaki. Dodatkowo wykorzystywany jest znak wyrównania '='.

Kodowanie przebiega poprzez podzielenie wejściowego ciągu danych $a$ o długości $l_a$ na sześciobitowe elementy $a'$ w ilości $l_{a'}$. Ciąg wyjściowy $b$ uzyskuje się poprzez wybranie $A[a'_i]$ dla $i \in [0, l_{a'})$.

Jeśli ciąg $a$ nie rozkłada się w całości na sześciobitowe elementy, tj. 6 nie jest dzielnikiem $l_a$, ciąg $a$ rozszerzany jest pustymi bajtami, dopóki taki rozkład nie będzie możliwy (jeden lub dwa bajty), a ciąg $b$ uzupełniany jest taką samą ilością znaków wyrównania.

Łatwo teraz wyliczyć wartość $l_{a'}$:
\begin{equation}
 l_{a'} = 4 * \lceil l_a / 3 \rceil
\end{equation}

Kodowanie to umożliwia bezpieczne przesyłanie danych binarnych za pomocą protokołów przeznaczonych dla transmisji tekstu z niewielkim narzutem obliczeniowym oraz objętościowym.

Dekodowanie przebiega poprzez wykonanie algorytmu w tył.
