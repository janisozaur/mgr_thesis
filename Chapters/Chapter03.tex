%************************************************
\myChapter{Aktualny stan zagadnienia}\label{ch:current_state}
%************************************************

Zaczynając w 2010r. swoją pracę inżynierską, jedynym dostępnym powszechnie systemem ,,alternatywnym'' systemem interakcji z komputerami, rozumianym tutaj w szerokim kontekście - t.j. także konsolami, był Wiimote firmy Nintendo - kontroler wyposażony w kamerę rejestrującą pozycję dwóch markerów świecących w podczerwieni, akcelerometr MEMS, a w póżniejszych wersjach także żyroskop MEMS. Od tego czasu, na skutek bardzo dobrego przyjęcia się takiego sposobu sterowania, szczególnie w grupie ,,\textit{casual gamers}'', nastąpił gwałtowny postęp w dziedzinie interakcji człowiek-komputer \cite{hcibook,reachout,movtar,behuman}.\\

\section{Konsole}

Dzisiaj producenci konsol prześcigają się w proponowanych technologiach wykorzystywanych przede wszystkim na potrzeby rozrywki urządzeniach wejściowych:
\begin{itemize}
 \item \textsmaller{Sony PlayStation Move}, 2H2010, wykorzystuje kamerę oraz kontroler z kulą podświetlaną diodą RGB, żyroskopem, magnetometrem oraz akcelerometrem (w sumie 9 stopni swobody), kamera zawiera 4-komorowy mikrofon, który dzięki analizie czasu odebrania dźwięku, umożliwia lokalizację źródła dźwięku,
 \item \textsmaller{Microsoft Kinect}, 2H2010, oparty o dwie kamery: jedną pracującą w paśmie światła widzialnego, drugą pracującą w podczerwieni, która odczytując wzorzec świetlny nadany przez znajdujący się w urządzeniu laserowy projektor, dostarcza mapę głębokości sceny. Oprogramowanie konsoli Xbox 360 przetwarza dostarczone dane i odtwarza z nich szkielet gracza,
 \item \textsmaller{Sony PlayStation Vita}, 1H2012, oferuje bogaty zestaw urządzeń wejściowych: żyroskop, akcelerometr, odbiornik GPS, ekran dotykowy (z przodu), panel dotykowy (z tyłu), dwie kamery,
 \item \textsmaller{Nintendo Wii U}, 2H2012, kontrolery tej konsoli zawierają dodatkowy wyświetlacz, moduł NFC\graffito{Near-field communications}, mikrofon, żyroskop, magnetometr, akcelerometr,
 \item \textsmaller{Microsoft Kinect (Xbox One)}, 2H2013, usprawniona wersja urządzenia \textsmaller{Kinect}, zawierająca kamerę type ,,time-of-flight'' badającą głębokość całej sceny w jednym przebiegu
 \item \textsmaller{Sony PlayStation 4}, 2H2013, kolejna wersja sprzętu opartego o \textsmaller{PlayStation Move} będzie zawierać dwie kamery, umożliwiając rekonstrukcję sceny w 3D, kontrolery zawierają  podświetlaną ściankę (w celu umożliwienia śledzenia ich przez kamery) oraz panel dotykowy.
\end{itemize}

Większość z wymienionych powyżej systemów opartych jest o podstawowe kontrolery zawierające przyciski, gałki, silniczki wibrujące oraz głośniki (lub możliwość podłączenia zestawu słuchawkowego), umożliwiające przesyłanie sygnałów zwrotnych z komputera dla użytkownika.

Nie należy zapominać o urządzeniach pracujących pod kontrolą systemów iOS oraz Android, których popularyzacja znacząco wpływa na dostępność urządzeń dotykowych, często z mnogością innych czujników.\\

\section{Kickstarter}

Nie tylko istniejący już producenci sprzętu rozwijają nowe technologie, od czasu powstania serwisu Kickstarter, platformy umożliwiającej finansowanie dużych przedsięwzięć osobom prywatnym przez tzw.\ \textit{crowd-funding}, z jego pomocą światło dzienne ujrzało kilka innowacyjnych projektów:
\begin{itemize}
 \item \textsmaller{Leap Motion}, kontroler wykorzystujący dwie kamery oraz diody podczerwone do oświetlania ,,sceny'', dostarczany z oprogramowaniem umożliwiającym śledzenie ruchów rąk,
 \item \textsmaller{Oculus Rift}, HMD\graffito{Head-mounted Display} z zestawem czujników umożliwiający swobodne rozglądanie się w świecie gry,
 \item \textsmaller{Mycestro}, zakładane na palec urządzenie zastępujące mysz, wykorzystujące 3-osiowy żyroskop oraz czyjnik dotykowy,
 \item \textsmaller{TouchKeys}, nakładki dotykowe na klawisze pianina.\\
\end{itemize}

\section{Inne}

Innym interesującym projektem, którego nie można uwzględnić w powyższych listach jest \textsmaller{Myo}, opaska zakładana na rękę odczytująca sygnały elektryczne płynące do mięśni, wykrywająca ruchy ,,u źródła''.\\

Wymienione powyżej, to tylko niektóre ze sfinansowanych projektów, pokazują jednak one trend dążący do usprawnienia metod komunikacji z komputerem. Cechą wspólną ich wszystkich jest usunięcie bariery, jaką stanowi wprowadzanie danych do urządzeń cyfrowych - sprawienie, aby komputery stały się niedostrzegalne dla użytkownika, który podczas codziennych czynności nie będzie zastanawiał się w jaki sposób zmusić komputer do działania. Odpowiedzialność zostanie przełożona na komputer, którego zadaniem będzie interpretacja zamiarów człowieka i wspomożenie go wtedy, gdy zajdzie potrzeba.\\

Znaczący postęp w tej dziedzinie, jaki miał miejsce na przestrzeni ostatnich kilku lat, oraz związane z nim zwiększenie popularności tego typu urządzeń oraz wsparcia ze strony sprzętu i oprogramowania zachęca do podejmowania dalszych prób, eksperymentów, które można już wykonywać w zaciszu domowym.
