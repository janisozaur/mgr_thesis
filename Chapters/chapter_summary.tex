%*****************************************
\chapter{Podsumowanie}\label{ch:summary}
%*****************************************

Celem pracy było stworzenie systemu pozwalającego na interakcję z~komputerem poprzez śledzenie ruchu palców.

Chociaż napisane w tym celu oprogramowanie jest sprawne i wypełnia postawione założenia, część sprzętowa wskutek napotkanych podczas realizacji nieprzewidzianych ograniczeń nie dostarcza danych wystarczającej jakości.

Próby obejścia występujących ograniczeń poprzez stopniowe, iteracyjne ulepszanie sprzętu poprawiało osiągane rezultaty, nadal jednak nie na tyle, aby usunąć z danych wszystkie szumy.

Praca nad tym systemem stanowiła kopalnię doświadczenia z zakresu zainteresowań, pozostawia jednocześnie furtkę do dalszego rozwoju.
W miarę dostępnego czasu i funduszy, możliwe przyszłe kroki obejmują:

\begin{aenumerate}
 \item zastosowanie mocniejszych diod,
 \item zastosowanie bardziej selektywnych fotodiod,
 \item zastosowanie filtra podczerwonego,
 \item wykorzystanie diod i fotodiod pracujących w spektrum bardziej oddalonym od światła widzialnego.
\end{aenumerate}

Po osiągnięciu danych o dostatecznie wysokiej wartości współczynnika sygnału do szumu możliwą ścieżką rozwoju oprogramowania jest próba przypisania jego części na mikrokontroler (w~celu odciążenia komputera) lub napisanie sterownika urządzenia wejściowego.
Możliwości te nie wykluczają się, idealnym wyjściem było by połączenie obu tych podejść.

System w swojej obecnej postaci nie mógłby konkurować z istniejącymi rozwiązaniami, jest jednak od nich różny w stopniu, który każe przypuszczać, że może znaleźć swoje zastosowanie.
