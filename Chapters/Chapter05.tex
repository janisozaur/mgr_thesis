%************************************************
\myChapter{Metoda}\label{ch:current_state}
%************************************************

W tym rozdziale przedstawię metodę wykorzystaną do określania pozycji obiektów znajdujących się w ramce.\\

Ramka składa się z dwudziestu modułów, każdy z nich zawiera jedną diodę LED oraz 8 fotodiod.

Podczas działania urządzenia zapalane są kolejno moduły ponumerowane od 0 do 19, układ jest zaprojektowany w taki sposób, aby w dowolnej chwili świecił się najwyżej jeden moduł. W czasie świecenia wybierane są moduły leżące po przeciwnej stronie i odczytywany jest ich stan, który następnie trafi do komputera. Host decyduje o tym które moduły należy zapalić, odpytać oraz w jakiej kolejności to zrobić. Mikrokontroler jest jednostką wykonawczą tych poleceń i dostarcza z powrotem dane w postaci wygodnej do analizy.

Na potrzeby pracy przyjmijmy, że światło z nadajnika rozchodzi się w postaci dyskretnych promieni \pauza wiązek światła łączących go z odbiornikami. Pozwala to na uproszczenie opisu metody działania bez poświęcania dokładności \pauza światło, które nie trafia w aktywną w danej chwili fotodiodę nie jest brane pod uwagę.

Przerwanie któregokolwiek z takich promieni poprzez zasłonięcie odbiornika powoduje zmianę stanu na jego wyjściu. Fotodiody oświetlone dają na wyjściu stan niski, zaś nieoświetlone \ppauza wysoki.

