%*****************************************
\chapter{Cel i założenia pracy}\label{ch:purpose}
%*****************************************

\section{Cel}

Celem pracy jest stworzenie, zademonstrowanie i ewaluacja systemu śledzenia palców przy wykorzystaniu technologi IR\graffito{IR, infrared \ppauza podczerwień.}.

W ostatnich latach przystępność komputerów znacznie wzrosła, a w ślad za tym poszedł rozwój systemów interakcji z urządzeniami cyfrowymi. Z każdą wersją sprzętu, czasami też oprogramowania, wprowadzane są coraz nowsze i nowsze metody wejściowe zacierające granice pomiędzy czynnościami związanymi z obsługą komputera, a naturalnymi ruchami.

Prym na tym polu wiodą konsole wraz z telefonami, czyli urządzenia nastawione przede wszystkim na dostarczanie rozrywki. Zaraz za nimi, skupiając się na innych aspektach, podążają metody sterowania robotami: wojskowymi, medycznymi, osobistymi. Chociaż rozwiązania z jednej dziedziny swobodnie przepływają do drugiej, to dla zwykłego użytkownika komputerów zmieniło się bardzo niewiele.

Tworzony system ma za zadanie zbadać możliwość przystosowania jednej z dostępnych metod detekcji obiektów na potrzeby komputera klasy PC.\\

\section{Założenia}

Tworzony system będzie składał się z dwóch części:
\begin{itemize}
 \item sprzętowej,
 \item programowej.
\end{itemize}

Obie te części będą ściśle pracować w tandemie, a ich przydatność osobno będzie ograniczonej wartości.

Pożądane jest, aby system pozwalał na śledzenie ruchu palców, co będzie eliminować konieczność wykorzystywania dodatkowych obiektów lub specjalizowanych narzędzi.

System ten jest znacząco różny od powszechnie dostępnych na chwilę obecną rozwiązań.\\

Praca ta stanowi także w pewnym sensie kontynuację mojej pracy inżynierskiej: ,,System śledzenia ruchów 6DOF''.
Traktuje ona o zagadnieniu z tej samej dziedziny, łączy je także fakt skonstruowania sprzętu na potrzeby każdej z nich.

Podobieństwo tych prac pozwala wykorzystać nabyte wtedy doświadczenie w praktyce.
