%************************************************
\myChapter{Testy}\label{ch:tests}
%************************************************

W tym rozdziale zaprezentuję metody i wyniki testów.

Testy dotyczą działania algorytmu wykorzystując przeszkody wirtualne, wygenerowane w programie oraz prawdziwe, o których informacja dostarczana jest z urządzenia. Przedstawione są też możliwości i ograniczenia sprzętu.\\

Z uwagi na konstrukcję, rozdzielczość sprzętu jest zmienna w zależności od miejsca. Dwuwymiarowy rozkład dokładności prezentujący ilość promieni przechodzących przez kwadraty o zadanych bokach można uzyskać wykorzystując ulepszony algorytm heatmapy, w czasie gdy ramka raportuje, że jest pusta.

Wspomniany rozkład dla $a = 5$ prezentuje rysunek~\ref{fig:scene_heatmap2_resolution_5}. Skala licznika $C$ jest następująca: $C_{min} = -15$, zaznaczone kolorem ciemnoniebieskim, $C_{max} = -1$, zaznaczone kolorem czerwonym. Przez pojedynczych białe kwadraty nie przechodzi żaden promień.

\begin{figure}
 \centering
 \makebox[\textwidth][r]{
  \resizebox{.9\largefigure}{!}{
    \def\svgwidth{0.9\largefigure}
    \includesvg{gfx/scene_heatmap2_resolution_5}
  }
 }
 \caption{Rozkład dokładności sprzętu, $a = 5$.}
 \label{fig:scene_heatmap2_resolution_5}
\end{figure}

Łatwo dostrzec, że w pobliżu krańców ramki dokładność gwałtownie spada. Dzieje się tak ponieważ przekroczona została granica ostatniego nadajnika, co znaczy, że ilość możliwych do zasłonięcia promieni staje się bardzo mała. W praktyce, ponieważ moduły nie dolegają ściśle do granicy ramki, dokładność w tych miejscach może być jeszcze niższa.

Ponieważ jednak ramka projektowana była pod kątem posiadanego monitora 24'', obszar ten jest z założenia nieużywany, gdyż zajmuje go obudowa wyświetlacza.\\

Jednym z pierwszych napotkanych problemów podczas oprogramowywania systemu był przypadek takiego umieszczenia obiektu, kiedy byłby on zasłonięty przez inny obiekt w scenie. Promienie zatrzymujące się  na pierwszym z nich, nie byłyby w stanie oświetlić drugiego z nich, w efekcie czego drugi pozostając w ,,cieniu'' pierwszego, nie był by wykryty. Sytuacja taka w przypadku jedynie dwóch obiektów ma nikłe szanse zaistnienia \pauza dla dostatecznie dużych odległości pomiędzy obiektami zawsze będą istnieć promienie niezasłonięte przez pierwszy obiekt.

Problem zaczyna nabierać większego znaczenia, gdy do trzech istniejących już obiektów, dodany zostanie czwarty i rozmieszczone będą na planie prostokąta.

Jak pokazują rysunki \ref{fig:scene_heatmap_5_rect_case} i \ref{fig:scene_heatmap2_5_rect_case}, pierwsza wersja algorytmu heatmapy słabo radzi sobie z tak postawionym zadaniem, jednak usprawniona wersja jest w stanie poprawnie wykryć wszystkie cztery przeszkody.

\begin{figure}
 \centering
 \makebox[\textwidth][r]{
  \resizebox{.9\largefigure}{!}{
    \def\svgwidth{0.9\largefigure}
    \includesvg{gfx/scene_heatmap_5_rect_case}
  }
 }
 \caption{Przypadek czterech przeszkód, pierwotny algorytm, $a = 5$.}
 \label{fig:scene_heatmap_5_rect_case}
\end{figure}

\begin{figure}
 \centering
 \makebox[\textwidth][l]{
  \resizebox{.9\largefigure}{!}{
    \def\svgwidth{0.9\largefigure}
    \includesvg{gfx/scene_heatmap2_5_rect_case}
  }
 }
 \caption{Przypadek czterech przeszkód, ulepszony algorytm, $a = 5$.}
 \label{fig:scene_heatmap2_5_rect_case}
\end{figure}

Testy sprzętu miały wiele faz, wymuszały też ewolucję projektu.\\

Pierwotna wersja zakładała dwie diody nadawcze, zamontowane powyżej płaszczyzny odbiorników. Miało to na celu wyeliminowanie problemu odbić odczytywanych w obiornikach, na które diody te nie świeciły.

W praktyce okazało się to zbędnym zabezpieczeniem, gdyż moduły mogące odczytać takie odbicie znajdowały się po tej samej stronie, co nadajnik, a nie były one uwzględniane w odczytach.

Pozwoliło to na zamontowanie nadajników bliżej płaszczyzny odbiorników, zwiększając jednocześnie szansę poprawnego wykrycia sygnału przez bardziej bezpośrednie ich oświetlanie.\\



Ze względu na ręczne lutowanie elementów modułów, pomimo zachowania najwyższej precyzji, nie wszystkie powstały jednakowe.
W przypadku przedniej strony ma to istotne znaczenie, ponieważ za krótka fotodioda schowana będzie w wywierconym na nią otworze, co ogranicza ilość dochodzących niej sygnałów.

Znacznym problemem okazała się histereza fotodiod, objawiająca się poprzez przełączenie jej stanu podczas zasłonięcia, jednak po odsłonięciu niektóre z nich nie powracały do spodziewanego, pierwotnego stanu.

Dodatkowo, fotodiody okazują się być podatne na światło dzienne, co w znacznym stopniu ograniczało możliwości testowania sprzętu za dnia lub w niezaciemnionym pomieszczeniu. Światło sztuczne, chociaż nadal powodujące przekłamania, daje znacznie bardziej przewidywalne rezultaty.

Zastosowane w modułach elementy, w szczególności bufory, powodują wydłużenie czasu propagacji sygnału z fotodiod, czyniąc odczytanie stanu magistrali bezpośrednio po wyborze odczytywanego modułu bezwartościowym.
W celu zniwelowania tego problemu dodane zostało opóźnienie pomiędzy wyborem modułu, a odczytaniem jego stanu.
Protokół komunikacji umożliwia modyfikację wartości tego opóźnienia (\ref{sec:commands} \nameref{sec:commands}, komenda \textsl{d}), jednak nie jest ona dostępna w programie \nameref{sec:raydisplay} (\ref{sec:raydisplay}).
Wielkość ta jest podawana w mikrosekundach, a domyślna wartość wynosi 100, została ona uznana za wystarczającą na podstawie serii testów.\\



Początkowo kalibracja urządzenia dokonywana była na mikrokontrolerze po uruchomieniu.
Rodziło to szereg problemów spowodowanych przez niestabilność danych, których nie dało się w prosty sposób ominąć.
Dodatkowym problemem była synchronizacja tak otrzymanej kalibracji z hostem.\\


Zmiana diod na mocniejsze, TSAL6400, zaadresowała w różnym stopniu listę powyższych mankamentów.

Większa moc powoduje większą pewność poprawnego odczytu danych przez odbiorniki, a także szybszą ich reakcję.

Dostępność dokładnej specyfikacji tych diod pozwoliła przeprojektować kalibrację na prostszą,a jednocześnie dającą więcej możliwości wersję ,,statyczną'', tj. dostarczaną w pliku wraz z programem.
Diody te świacą w promieniu 25$^{\circ}$ (podawane jest wychylenie w jedną stronę od osi diody), co po pomiarach oznacza możliwość oświetlenia modułów naprzeciwko oraz po jednym na lewo i prawo od niego.
Dostarczana domyślna kalibracja zakłada dokładnie takie parametry, przy czym nadajniki brzegowe odczytują tylko dwa moduły: moduł naprzeciwko oraz sąsiadujący z nim po tej samej stronie ramki.\\

Fotodiody nadal podatne są na światło dzienne oraz (w mniejszym stopniu) sztuczne, a także nadal zwracają zaszumione dane.

Przykładowy zapis pustej sceny w różnych warunkach oświetlenia widać na rysunku \ref{fig:scene_empty_1} i \ref{fig:scene_empty_2}.\\

Rysunki \ref{fig:scene_with_one_obstacle_1} i \ref{fig:scene_with_one_obstacle_2} pokazują scenę z jednym obiektem (jego pozycja została ręcznie oznaczona) w różnych warunkach oświetlenia.\\

\begin{figure}
 \centering
 \makebox[\textwidth][l]{
  \resizebox{.9\largefigure}{!}{
    \def\svgwidth{0.9\largefigure}
    \includesvg{gfx/scene_empty_1}
  }
 }
 \caption{Pusta scena 1}
 \label{fig:scene_empty_1}
\end{figure}

\begin{figure}
 \centering
 \makebox[\textwidth][l]{
  \resizebox{.9\largefigure}{!}{
    \def\svgwidth{0.9\largefigure}
    \includesvg{gfx/scene_empty_2}
  }
 }
 \caption{Pusta scena 2}
 \label{fig:scene_empty_2}
\end{figure}

\begin{figure}
 \centering
 \makebox[\textwidth][r]{
  \resizebox{.9\largefigure}{!}{
    \def\svgwidth{0.9\largefigure}
    \includesvg{gfx/scene_with_one_obstacle_1}
  }
 }
 \caption{Scena z jednym obiektem (zaznaczonym ręcznie) 1}
 \label{fig:scene_with_one_obstacle_1}
\end{figure}

\begin{figure}
 \centering
 \makebox[\textwidth][r]{
  \resizebox{.9\largefigure}{!}{
    \def\svgwidth{0.9\largefigure}
    \includesvg{gfx/scene_with_one_obstacle_2}
  }
 }
 \caption{Scena z jednym obiektem (zaznaczonym ręcznie) 2}
 \label{fig:scene_with_one_obstacle_2}
\end{figure}

\section{Wyniki}

Przedstawione wcześniej rezultaty testów pokazują zachowanie algorytmu i sprzętu w różnych przypadkach.

W szczególności rysunki \ref{fig:scene_empty_1}, \ref{fig:scene_empty_2}, \ref{fig:scene_with_one_obstacle_1} i \ref{fig:scene_with_one_obstacle_2} prezentują możliwą do uzyskania dokładność.

Pierwsza para obrazuje poziom szumów w zwracanych danych.
Jest on miejscami dostatecznie duży, aby mógł zostać pomylony z~prawdziwym obiektem znajdującym się w ramce.\\

Druga para obrazków pokazuje wielkość obiektu, jaki potrzebny był do testów.

Obiekty mniejsze, a w szczególności palce były dla systemu ledwo widoczne lub zupełnie niewidoczne.

Nawet w przypadku tak dużych obiektów, zwracane dane nie były zadowalającej jakości, posiadały wiele przekłamań, które ograniczały możliwość prawidłowego wykrycia konturu przeszkody.\\

Zaimplementowany algorytm mógłby dawać lepsze rezultaty, gdyby nie znaczny poziom szumu danych wejściowych.

Jest on jednocześnie prosty w implementacji, oparty o proste działania, musi jednak przetwarzać duże ilości danych (w zależności od ustawionych opcji).

