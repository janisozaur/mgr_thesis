%************************************************
\myChapter{Testy}\label{ch:tests}
%************************************************

W tym rozdziale zaprezentuję metody i wyniki testów.

Testy dotyczą działania algorytmu wykorzystując przeszkody wirtualne, wygenerowane w programie oraz prawdziwe, o których informacja dostarczana jest z urządzenia. Przedstawione są też możliwości i ograniczenia sprzętu.\\

Z uwagi na konstrukcję sprzętu, rozdzielczość sprzętu jest zmienna w zależności od miejsca. Dwuwymiarowy rozkład dokładności przentujący ilość przechodzących przez kwadrat o zadanym boku można uzyskać wykorzystując ulepszone algorytm heatmapy podczas gdy ramka zwraca same zera \pauza może być pusta lub wyłączona z prądu.

Wspomniany rozkład dla $a = 5$ prezentuje rysunek~\ref{fig:scene_heatmap2_resolution_5}. Skala licznika $C$ jest następująca: $C_{min} = -15$, zaznaczone kolorem ciemnoniebieskim, $C_{max} = -1$, zaznaczone kolorem czerwonym. Przez pojedynczych białe kwadraty nie przechodzi żaden promień.

\begin{figure}
 \centering
 \makebox[\textwidth][r]{
  \resizebox{.9\largefigure}{!}{
    \def\svgwidth{0.9\largefigure}
    \includesvg{gfx/scene_heatmap2_resolution_5}
  }
 }
 \caption{Rozkład dokładności sprzętu, $a = 5$.}
 \label{fig:scene_heatmap2_resolution_5}
\end{figure}

Łatwo dostrzec, że w pobliżu krańców ramki dokładność gwałtownie spada. Dzieje się tak ponieważ przekroczona została granica ostatniego nadajnika, co znaczy, że ilość możliwych do zasłonięcia promieni staje się bardzo mała. W praktyce, ponieważ moduły nie dolegają ściśle do granicy ramki, dokładność w tych miejscach może być jeszcze niższa.

Ponieważ jednak ramka projektowana była pod kątem posiadanego monitora 24'', obszar ten jest z założenia nieużywany, gdyż zajmuje go obudowa wyświetlacza.\\

Jednym z pierwszych napotkanych problemów podczas oprogramowywania systemu był przypadek takiego umieszczenia obiektu, kiedy byłby on zasłonięty przez inny obiekt w scenie. Promienie zatrzymujące się  na pierwszym z nich, nie byłyby w stanie oświetlić drugiego z nich, w efekcie czego drugi pozostając w ,,cieniu'' pierwszego, nie był by wykryty. Sytuacja taka w przypadku jedynie dwóch obiektów ma nikłe szanse zaistnienia \pauza dla dostatecznie dużych odległości pomiędzy obiektami zawsze będą istnieć promienie niezasłonięte przez pierwszy obiekt.

Problem zaczyna nabierać większego znaczenia, gdy do trzech istniejących już obiektów, dodany zostanie czwarty i rozmieszczone będą na planie prostokąta.

Jak pokazują rysunki \ref{fig:scene_heatmap_5_rect_case} i \ref{fig:scene_heatmap2_5_rect_case}, pierwsza wersja algorytmu heatmapy słabo radzi sobie z tak postawionym zadaniem, jednak usprawniona wersja jest w stanie poprawnie wykryć wszystkie cztery przeszkody.

\begin{figure}
 \centering
 \makebox[\textwidth][r]{
  \resizebox{.9\largefigure}{!}{
    \def\svgwidth{0.9\largefigure}
    \includesvg{gfx/scene_heatmap_5_rect_case}
  }
 }
 \caption{Przypadek czterech przeszkód, pierwotny algorytm, $a = 5$.}
 \label{fig:scene_heatmap_5_rect_case}
\end{figure}

\begin{figure}
 \centering
 \makebox[\textwidth][r]{
  \resizebox{.9\largefigure}{!}{
    \def\svgwidth{0.9\largefigure}
    \includesvg{gfx/scene_heatmap2_5_rect_case}
  }
 }
 \caption{Przypadek czterech przeszkód, ulepszony algorytm, $a = 5$.}
 \label{fig:scene_heatmap2_5_rect_case}
\end{figure}